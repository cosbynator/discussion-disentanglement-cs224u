\documentclass[10pt]{article}
\usepackage{acl2012}
\usepackage[margin=1.0in]{geometry}
\usepackage{times}
\usepackage{latexsym}
\usepackage{amsmath}
\usepackage{multirow}
\usepackage{url}


\DeclareMathOperator*{\argmax}{arg\,max}
\setlength\titlebox{3.5cm}    % Expanding the titlebox

\title{{\small CS224U Final Project} \\ Discourse Disentanglement \\{\small Project Milestone}}
\author{Julius Cheng \\
  \\\And
  Thomas Dimson  \\
  \\\And
  Milind Ganjoo \\
}
\date{}
\begin{document}
\maketitle

\section{Overview and Goals}
Our project focuses on conversation disentanglement, which broadly refers to
the problem of guessing the structure of a multi-party, asynchronous
conversation. In particular, we focus on the task of reconstructing conversation
trees of modern internet forums like \textit{Reddit} from a linear view. Our unit of 
analysis a time-ordered list of messages and our output in a candidate thread tree. 
With that in mind, we wish to answer the following questions:

\begin{itemize}
  \item To what degree does semantic knowledge allow a machine to understand the 
    flow of online forum discussions?
  \item How much does non-semantic information, such as timestamps and referencing
  other speakers, aid the system?
  \item Does performance vary across domains? Do factors such as tree depth,
    length of messages, and similarity of messages affect our ability to
    reconstruct conversations? 
\end{itemize}

In the interests of academic honesty, we have made our git repository public at
\url{https://github.com/cosbynator/discussion-disentanglement-cs224u}

\section{Previous Approaches}
\label{sec:approaches}
As discussed in our literature review, there has been some limited prior 
work in reconstructing threaded discussion in online conversions. 

Work such as \cite{Elsner2008a} 
uses internet chat data and attempts to cluster messages into coherent 
conversations using a MaxEnt classifier. In this case, data has to be hand annotated and
the gold set is subjective.

Closer to our project was the work of \cite{Aumayr2011a}. Here, authors took
internet forum posts from traditional \texttt{vBulletin} style message boards
and attempted to reconstruct the reply structure from a linearized view. The authors
use a decision tree with features such as TF-IDF distance and elapsed time. The biggest
difference between them and us is that they use a reply graph as an evaluation metric,
where vertices are users and edges indicate existence of replies. We evaluate correctness
per-message rather than per-user.

Particularly interesting is the work of \cite{Wang2011a} which used a conditional random
field to reconstruct newsgroup-style conversations. This work has a treasure trove of
interesting features, and we anticipate implementing many of them in the upcoming weeks.
Furthermore, time permitting, we want to see if a conditional random field is a good way
to model our problem and improve performance.

{Milind - want to add anything here? What data did they use?}

\section{Current Approach}
\label{sec:approach}
We are approaching conversation reconstruction as a \textit{classification} problem
between parent and children messages. We take a threaded conversations from a modern internet forum 
(described in Section~\ref{section:dataset}, flatten them into a time-sorted list, and them feed
them into our reconstruction \textit{classifier} to make decisions on which posts are in reply
to one another. The details our reconstruction algorithm are described in Section \ref{sec:classifier}.

{Probably more here?}

\section{Progress}
Macroscopically, we have
been successful in creating a pipeline to reconstruct conversation trees and
evaluate our performance based on two metrics. The numbers leave something to be desired and
we anticipate spending most of our remaining time on feature engineering and
improvements to our classifier.

\subsection{Dataset}
\label{section:dataset}
Our approach differs from most of those listed in Section \ref{sec:approaches}
by the scale of our approach: we are trying to reconstruct complex coversations
with large branching numbers.  In that vein, we extract data from Reddit, an
online link-sharing and discussion site. To ensure all conversations have a root
element, we extract conversations from AskReddit, a popular subforum where 
users initiate topics of conversation with a post. We treat the
initial post as the root of a conversation tree, and each top-level comment as a
child of the root. Each message node is annotated with various data such as
timestamp, username, and more.

To grab the data, we wrote a quick JRuby crawler and took a small sample (342) of the top
AskReddit posts over the last month. The crawler was written to make data gathering easy
and we anticipate creating a bigger dataset to report our final results. Table \ref{table:stats}
shows some aggregated statistics of the preliminary crawl. In particular, we have an average
of 200 messages per tree and a branching factor within a rounding error of 1.0. The root
node has an average branching factor of 55.50, suggesting that discussions are wide while
sub-discussions tend to be narrow. It is also important to note that standard
deviation for many of our statistics are often quite large relative to the
value, suggesting the wide variance in conversations on reddit.

\begin{table}[h]\footnotesize
 \begin{tabular}{| l | l | l |} 
   \hline
   \textbf{Statistic} & \textbf{$n / \mu$} & \textbf{$\sigma$} \\
   \hline
    \# Conversation Trees &  342 & \\
    \# Conversation Nodes (Posts) & 68639 & \\
    \# Nodes Per Tree &  200.70 & 200.20 \\
    Mean Branching Factor & 1.00 & 4.20 \\
    Mean Root Branching Factor &  55.50 & 57.91 \\
    Mean Node Depth  & 3.56 & 3.40 \\
    Mean Parent Lag & 7767.96s & 11331.07s \\
    Mean Body Length & 200.43 c & 483.07 \\
   \hline
  \end{tabular}
  \caption{Data statistics of our preliminary conversation dataset}
  \label{table:stats}
\end{table}

We have divided the input data randomly into training / development / test sets
at a 70\% / 10\% / 20\% split.

\subsection{Evaluation}
For now, we are primarily considering F1 score as our evaluation metric. We
compute precision and recall based on how many of our guessed $(parent,child)$
conversation edges match the gold tree. We are considering two
versions of F1: one were precision and recall is calculated pairwise across all
messages in all conversations, and another where F1 is calculated
per-conversation, and the scores are averaged in the end. We expect to have
very poor performance in large conversations, so we expect the average F1 score metric to be
more forgiving. Both metrics have been implemented in our code.

Noting that it can be more important to get lower-depth subtrees correct, we
are also considering weighting a clustering metric such as $B^3$. Here, we can
treat subtrees at a given depth as clusters, comparing against gold clusters at
that depth. We could weight proportional to the inverse depth to reward us for
capturing more of the general discourse with less sensitivity to individual placement.

\subsection{Classifier and Preliminary Results}
\label{sec:classifier}
{Describe how we take highest confidence previous tree}
These trees are given to the learner as training data. One can say that
pairs of messages where one is in reply to the other are positive training
examples, and all other pairs are negative examples.

At test time, the learner is given a conversation linearized by time, with all
reply structure removed, and it outputs a predicted tree. Our classifier does this
by classifying, for each pair of messages, $M1$ and $M2$, whether $M2$ is in reply
to $M1$. We deviate from a normal classification problem, because each message can
be in reply to only one message, one that occurs at a previous time. Therefore, rather
than classifying whether each pair of messages are linked, for each message, we estimate
probabilities that it is linked to each message previous in time. We predict the
message with the highest probability to be the parent.

{add explanation of classifier and features}

These results are a severe but unsatisfactory improvement over two trivial baseline
systems: one guesses that all messages are made in reply to the root, and the
other guesses that all messages are made in reply to most recent previous message.

\begin{table}[h]\footnotesize
 \begin{tabular}{| l | l | l |} 
   \hline
   \textbf{Statistic} & Pairwise F1 & Average-tree F1 \\
   \hline
    SVM baseline &  & \\
    All-reply-to-root baseline & 0.278 & 0.278 \\
    One-thread baseline & 0.021 & 0.021 \\
   \hline
  \end{tabular}
  \caption{Data statistics of our preliminary conversation dataset}
  \label{table:stats}
\end{table}

Since our crawled dataset currently contains conversations of similar size, our
distinct evaluation metrics result in scores that are equal to the 4th decimal.

\section{Next Steps}
- Many many more features
- CRF models~\cite{Wang2011a}

In addition, have discussed trying a different data source (Hacker News) to
see how our performance varies accross domains. We anticipate Hacker News
to be a fairer representation of our classifier since discussions tend to be 
smaller and more on-topic.

\bibliography{milestone}{} 
\bibliographystyle{acl2012}

\end{document}
