\documentclass{article}
\usepackage{spverbatim}
\usepackage{fullpage}
\usepackage{amsmath}
\usepackage{listings}
\usepackage{color}

\setlength{\parskip}{1ex}
\setlength{\parindent}{0pt}

\definecolor{dkgreen}{rgb}{0,0.6,0}
\definecolor{gray}{rgb}{0.5,0.5,0.5}
\definecolor{mauve}{rgb}{0.58,0,0.82}

\title{CS224U Homework 11}
\author{
Julius Cheng\\
Thomas Dimson\\
Milind Ganjoo
}
\begin{document}
\maketitle

\section*{Intrinsic Evaluation}
Intrinsic evaluation measures the performance of a system
based on predefined criteria about the system itself. An example
might be the accuracy of word sense disambiguation against
a set of human-annotated ground truths. 

In our project, we are dealing with the problem of reconstructing
structure in data after it has been lost. Namely, we have threaded
conversations where we throw away the threading and attempt to 
reconstruct it using the content of the messages. We could evaluate
our system \textit{intrinsically} based on precision/recall of the 
(parent, child) links in each message.

\cite{Aumayr2011a} uses a similar evaluation. In their case,
they are dealing with constructing (parent, child) links in a traditional
online forum. They calculate Precision/Recall based on the number of 
links they correctly classify and compare their results against previous
papers and baselines using the blended F1-score.

\section*{}

\bibliography{hw11}{} 
\bibliographystyle{plain}

\end{document}
