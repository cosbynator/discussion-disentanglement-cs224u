\documentclass[11pt]{article}
\usepackage{acl2012}
\usepackage{times}
\usepackage{latexsym}
\usepackage{amsmath}
\usepackage{multirow}
\usepackage{url}
\DeclareMathOperator*{\argmax}{arg\,max}
\setlength\titlebox{6.5cm}    % Expanding the titlebox

\title{{\small CS224U Final Project} \\ Discourse Disentanglement \\Literature Review}
\author{Julius Cheng \\
  {\tt juliusc@stanford.edu}
  \\\And
  Thomas Dimson  \\
  {\tt tdimson@cs.stanford.edu}
  \\\AND
  Milind Ganjoo \\
  {\tt mganjoo@stanford.edu}
}
\date{}

\begin{document}
\maketitle

\section{Introduction}
Our project is focused on reconstructing threaded conversation trees
from a flattened representation. Intuitively, one could think of 
taking a post's comment tree (e.g. Reddit), flattening it, sorting by time, performing
hierarchical clustering using linguistic phenomena and measuring performance 
against the original comment trees. As we have
discovered, this project treads into the territory of \textit{discourse disentanglement}.
While our particular problem does not appear to have been tackled in the past,
there is a good amount of literature about disentangling flows of conversation
in traditional internet forums, IRC chats, Twitter and beyond.

\section{Literature Overview}
Related work to conversation disentanglement varies between the datasets used, the 
sub-problem addressed and the method of evaluation. We have grouped the literature into
three broad categories described below.

\subsection{Forum-style Disentanglement}
\label{sec:threaded}
We define ``forum-style disentanglement'' as work that deals creates arbitrary
graphical links between posts under a related topic. The primary characteristic of the problem 
is longer, relatively well formed posts linked by relatively unsubjective ground truth links.

In \cite{Wang2008a}, address reconstructing thread structure using using 
message similarity alone. They take a flat
time-ordered conversation and use a similarity matrix to add graphical
links between messages. They claim that flat conversation structure
limits further data analysis, such as identification of significant
contributions. 

The basis of their algorithm is message similarity through TF-IDF: they
construct a lower triangular cosine-distance matrix of posts and use rules 
for conversion into a graph adjacency matrix. Their baseline is thresholding the
matrix, which they augment by limiting the column-distance (window size), and discounting
the matrix by the time lag between elements. The authors use a grid search to
determine right value for the threshold and discount parameters.

\cite{Wang2008a}'s dataset, \texttt{LegSim}, consists of 28 topics with over 
478 messages . Each topic is a discussion about a hypothetical legislative bill, 
and the messages are written by students discussing the bill in a political science class. 
The data itself is flat, but the authors manually create directed links representing
parent-child relationships. It struck us as peculiar that the authors chose to work
with such a small corpus and rely on manually annotated ground truth instead of using
threaded ground truth.

Accordingly, the authors evaluate their predicted annotations using precision/recall of
their graph edges between messages. They identify their best results using
thresholding and a time-based discount for a seemingly-poor result of 0.40 
F1-score. It is unlikely that higher scores could be achieved without reaching
beyond text similarity. That said, the work may provide a good baseline system
for us to evaluate against. At the very least, we will consider a learning
system instead of a rule-based system and utilizing features besides post
similarity.  If nothing else, we need to use a dramatically larger corpus.

The graph-based similarity detection method described previously by
\cite{Wang2008a} utilizes both content similarity and temporal information to
predict reply relationships. However, the threshold parameters of the model are
manually tuned. In contrast, \cite{Wang2011a}, describes an approach based on conditional random
fields, which are able to utilize features for link prediction
without having to model the dependent distributions of features.

The \cite{Wang2011a} paper defines a conditional distribution over the reply
relationship given a post sequence and formulates it as a Maximum a Posteriori
(MAP) problem. A crucial component of this task is to define a set of features
that capture dependencies between posts. The paper describes two categories of
features. The first type of features depend only on attributes drawn from a pair
of posts, such as content similarity score, recency, author name reference, and
indicator variables that give special importance to posts made last or first in
a given thread. The second kind of features also take the parent assignments of
the posts into account, which allows scoring of characteristics like the
similarity between a post and the \emph{parent} of another post to determine
content propagation, or accord importance to posts that reply to the same author
and are also written by the same author (which is indicative of friends carrying
out a back-and-forth conversation in a thread). The weights for the full set of
13 features is then trained through a supervised classifier that maximizes the
conditional likelihood for the reply sequence.

The classifier-based conditional random fields approach is novel for the number
of interesting features it introduces, and it also opens up the possibility of
extending the existing framework by adding more features.

\cite{Wang2011a} also introduces the novel \emph{path accuracy} metric, which
calculates the proportion of correct paths from each node to the root in a
conversation tree. This approach, however, would work best for a threaded
conversation that begins from a single topic or node, and would probably not be
extensible to a flatter IRC-style conversation where multiple root topics might
be ongoing in parallel, which is another type of threaded conversation we would
like to analyze. Still, we imagine it possible to usefully average the path
accuracy for paths to multiple such nodes.

Similar to \cite{Wang2011a}, \cite{Aumayr2011a}'s attempts machine learning to 
tackle the problem of reconstructing conversation trees in a traditional
vBulletin-powered online forum. Here, each forum consists of a \textit{board}, which
contains many \textit{threads}, each of which contains a series of
\textit{posts} by users. Unlike \cite{Wang2008a}, the motivation for their work
comes from reconstructing reply structure even after it has been lost in the
source database. This assumes that there was thread-structure in the original
data.

The authors use a learned classifier to classify pairs of posts as (parent,
reply) or other. Their features are a mix of shallow textual and post
metadata features: TF-IDF weighted text distance, presence of quotes and a
user's name, post-distance and time difference. In the end, they report their 
best results using a C4.5 decision tree. Since the classification is binary, it 
is unclear how the authors cope with ambiguity (e.g $(p1, r)$ and $(p2,r)$ both 
classified as reply pairs). Although post distance is their best feature, we are skeptical
that this will generalize to message boards that have a different ordering.

The authors collected data from two years of forum posts on
\url{boards.ie}. When browsing a \textit{thread}, users have a choice of either
pressing ``quote'' (i.e. replying to a post) or pressing ``reply'' (i.e.
replying to a thread). The choice is retained in the dataset and used in a
tree form as ground truth data. Without much justification, the authors eliminate
threads with too few or too many replies, and then split their data 50/50 into 
test and training to make 213,800 posts in total.

Their results are evaluated using precision and recall of individual
(post, reply) pairs. They report impressive results - an F1-score of 0.925 using their best
classifier. The results are contrasted to a baseline classifier that always
identifies the previous post as the parent of a reply, achieving 80\% F1-score. 
They also contrast their results with implementations from two other papers we looked at,
\cite{Elsner2008a} and \cite{Wang2008a}, achieving 88\% F1 and 44.4\% F1-score
respectively.

\subsection{Chat Disentanglement}
\label{sec:chat}
We identify the \textit{chat disentanglement} problem as attempting to group
shorter conversational-style messages into coherent sub-conversations. Like 
Section \ref{sec:threaded}, authors still create graphical links between messages. The major
difference is that the length of each message in a conversation is shorter, and ground truth
data is quite subjective (what constitutes a sub-conversation?).

\cite{Elsner2008a}  paper treats disentanglement as a clustering problem between 
utterances rather than considering speakers. The authors note that speaker-based clustering is
particularly problematic when conversation topics change.

The data for this problem is the manually labeled \#\#Linux corpus from an IRC
chat room. The annotation for each sentence places it in a certain conversation.
The corpus also contains information about schisms (change of topic) or minor
digressions. It is worth investigating whether this manual annotation scheme can
be applied to a different corpus, which is less topical and where the
demarcations between conversations may not be so clear. We anticipate the choice
of a more topical corpus would yield better classification accuracy.

The model first uses a MaxEnt classifier to decide whether a given pair of
utterances belongs to the same conversation. In general, the more likely case is
for them to be of different conversations. It then attempts to cluster the
utterances to maximize the weighted accuracy of the classifier. However, since
this is a correlation clustering problem (which is NP-complete), the paper uses
a greedy search approximation to the clustering process. This yields reasonable
results; however, we anticipate using this approach and its associated methods
as a baseline only.

The model for this paper also describes more features that could be applied to
our planned supervised learning model. These features include pause length
between utterances and word repetitions bucketed according to unigram
probability to indicate an ongoing conversation about something.

\cite{Mayfield2012a} presents a model for chat disentanglement that annotates three types of
information: 1) Negotiation labels, or whether an utterance is providing or
receiving information, 2) sequences, or exchanges of information, and 3)
threads, the outcome of the standard disentanglement problem.

The learning model is modularized along these lines to predict these classes of
annotation. The three classifiers mainly use n-gram features, including simple
term frequency and TF-IDF. Not much discussion is given on the choice of these
features, and why each used different ones. For example, the Negotiation
classifier uses unigrams and bigrams frequencies and part-of-speech tags, while
the Previous Cluster classifier, which predicts sequences, uses TF-IDF. 

The algorithm performs two passes over the data, one in which negotations and
sequences are generated, and the second in which sequences are grouped to form
threads. For sequence prediction, the model has five hard rules regarding the
structure of the Negotiation labels. For example, within the same sequence, the
provider of the information cannot be the same as the receiver. Also, in a
sequence, at most one continuous series of utterances may be made by the
information provider. These constraints are defined as an integer linear
programming problem and are used by the model to prevent classifications that
break domain rules. Interestingly, the human annotators of the data set are
expected to conform to these rules, guaranteeing that these rules are
“correct.”

The dataset used were weekly chat meetings for a cancer support group, with 45
conversations bucketed equally between 2 participants, 3-4, and at least 5. The
results were compared against baselines such as treating every message against
a new sequence. It also compared against results in negotiation labeling,
sequence labeling, and thread detection from a second set of human annotators,
performing better in some cases.

This study is very relevant to our project and introduces two major ideas to
consider. One is the idea of using knowledge exchange as a feature of
consideration in disentangling threads. The second is the two-pass model of
first constructing sequences and merging them into threads in the second pass.
In modeling what a thread is, it can be useful to define its components,
finding them, and merging them to produce threads. However, we find that the
data collection methodology requires significant expertise, given the task of
understanding deep linguistic concepts while adhering to logical domain
constraints.

\subsection{Speech/Dialogue Acts}
Some authors consider the problem of speech disentanglement as traversing 
a finite state automaton between different \textit{speech acts}, such as
``question'', ``answer'' or ``statement''. This is a classical problem in
natural language processing, but many authors evaluate their success extrinsically 
using discourse disentanglement.

\cite{Paula}, builds an unsupervised Hidden Markov Model that 
maps speech acts to hidden states. The idea is that the words of a message in a conversation are generated 
according to language models associated
with a state or speech act. 

While the use of Hidden Markov Models is not new, the model presented in this
paper allows a message to be in a \emph{mixture} of states rather than exactly
one class, as in most HMMs. In this sense it is similar to topic models such as
LDA that assume a distribution over multiple classes (topics).

The most exciting contribution of this paper is the discovery of hidden states
that actually correspond to speech acts, for example, a collection of
question-related pronouns and words corresponding to a speech act beginning a
forum post about a technical question. The revelation of the hidden structure of
a conversation motivates further investigation into clustering messages by the
speech acts they convey, as investigated in the next paper on Twitter
unsupervised modeling.

Like \cite{Paula}, \cite{Ritter2010a} considers unsupervised tragging of speech acts.
The paper describes an unsupervised LDA-like approach to act tagging 
and utilizes Twitter \textit{conversations} for
training purposes. The model was evaluated extrinsically by reconstructing 
the original order of a scrambled conversation. The author's describe their motivation as
generalizing act tagging to remove the human required in supervised approaches. 

The authors constructed a dataset of 1.3 million Twitter \textit{conversations}, 
where conversations are alternating series of reply-tweets between users. After filtering out shorter 
and longer conversations, the dataset was used for unsupervised training and evaluation.

\cite{Ritter2010a} tells a generative story of Twitter conversations where a sequence of dialogue
acts are chosen for each post, with words filled in by an act-specific language model.
Because is biased to model \textit{topics}
instead of \textit{acts}, the model is refined to include a topic 
for the entire conversation. The language model is also refined to choose words from
the conversation topic, the dialog act of the tweet, and from general English. The result
is a variant of LDA and much of the paper is dedicated to filling out the details of the model.

Most relevant to us are the two forms of evaluation they propose. For
qualitative evaluation they create a probabilistic finite automaton with
transitions between speech acts through a conversation. The speech acts
themselves are hand-labeled using the top-40 words in each act cluster. The 
authors admit this is a very subjective and domain-specific evaluation. As such,
they propose a quantitative evaluation method that quantifies the ability of
the model to predict the ordering of posts in a conversation. They took
1,000 randomly sampled conversations, computed all permutations
of each conversation and then chose the ordering that had the maximum
probability according to the model. They compared this ordering to the gold
ordering using Kendell $\tau$ distance. For a baseline they chose a bigram
model, the details of which are omitted from the paper. They report a ten-fold improvement
over the baseline, or about a 0.3 $\tau$-correlation. It is encouraging to know
that conversations appear to have a \textit{role} structure. We believe we can get a lot of mileage
by adding in additional features like \cite{Wang2008a}'s post-similarity.

Unlike other speech act papers, \cite{Kim2012} focuses on classifying utterances
in multi-party chat scenarios. The basic unit
of analysis here is the contiguous message written by an individual. The
experiment classifies messages into discourse acts such as a greeting, yes/no
question, request for information, or 11 other categories defined in previous
works, with two new ones introduced here.

\cite{Kim2012} defines several classes of features, then runs various classifiers on
combinations of them. The features include TD-IDF and boolean unigrams, with
and without stemming and lemmatization. They also learn on a special subset of
keywords that indicate various discourse acts, defined in previous works. Other
features are 1) structural attributes, such as length of message, location of
message within the dialogue, and user ID of the speaker, and 2) utterance
interactions, like the classification and content of previous messages, and
previous utterances by the author.

Their experiment uses Naive Bayes, SVM, and Conditional Random Fields on two
datasets, one topic-specific forum and one casual live chat. The best
performance in both datasets approaches 100% on a model using bag-of-words,
keywords, and previous utterances, but not structural information. The error
analysis shows that most errors pertain to ambiguity between pairs of dialogue
acts, like statements versus respsonses, and over certain words that appear
frequently in multiple dialogue acts. Finally, the authors conclude that the
failure of structural information compared to previous works is likely due to
entanglement.

While this work is not directly about entanglement, it produces good results in
datasets that are entangled. The authors conclude that work on
disentanglement is needed to improve their results, but we suggest that their
results show promise in the problem of disentanglement. If we can classify
dialogue acts without structural information, this could be a useful feature
for us. Some dialogue acts naturally follow other dialogue acts, so training a
model on which dialogue acts follow others could yield good results.
Additionally, this study shows that techniques applied to forums can work well
in live chat scenarios, and vice versa, which indicates that our future model
could potentially succeed in both settings.

\section{Conclusion}

We have surveyed nine studies on dialogue in multi-party entangled online 
chat or forum settings. We reviewed several experiments that directly address
the problem of disentangling conversation threads, either in long-form forum
or newsgroup messages or in short-form chat or Twitter conversations. These
invariably involve deconstructing messages into feature vectors and running a 
well-known machine learning algorithm such as HMM, CRF, and SVM over them.
There is some variety in how threads are constructed. \cite{Aumayr2011a}
a binary classification on whether two messages have a parent-child
relationship, while \cite{Mayfield2012a} cluster messages into threads with 
some cosine similarity metric. Our initial approach to the project will be to 
start with a model that emulates some of the above.

Beyond disentanglement, we reviewed three papers that discuss annotating 
speech acts. \cite{Ritter2010a} and \cite{Paula} use unsupervised clustering 
to discover speech acts, while \cite{Kim2012} use supervised learning on a set 
of pre-fabricated labels based on linguistic theory. Based on the findings of 
\cite{Kim2012}, we expect speech acts to be beneficial features, and we hope 
to discover a novel way to manually create or learn rules for sequencing 
utterances of various combinations of speech acts. Based on the findings of
\cite{Ritter2010a} and \cite{Paula}, we realize that it is possible to induce 
a set of speech acts during learning as to avoid using artificial categories 
and requiring additional human annotation.

We have found each of these papers to be relevant, useful, and inspirational
for our project on multi-party dialogue disentanglement.

\bibliography{lit_review}{} \bibliographystyle{acl2012}

\end{document}
